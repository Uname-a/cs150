\documentclass[10pt,a4paper]{report}
\usepackage[utf8]{inputenc}
\usepackage[margin=0.75in]{geometry}
\usepackage{mathtools}
\usepackage{siunitx}
\usepackage{amsfonts}
\usepackage{amssymb}
\author{matt baird}
\title{hello}
\begin{document}
\maketitle
\chapter{}
\section{units standards and the si system}
\paragraph{theories are created to explain observations and then tested basd on their predictions}
\chapter{Describing Motion: Kinematics in one dimension}
\paragraph{}
\section{referance frames and displacement}

\begin{itemize}
\item Any mesurement of position, distance, or speed must be made with respect to a referance frame. 
\item distinction between distance and displacement. 

\item displacement is how fare the object is from its starting point, reguardless of how it got there 
\item Distance traveled is mesureed along the actual path 
\item Mathmatically we wil represent position by the cordanite x
\item The  displacement is written   $ \bigtriangleup x = x_{2} - x_{1} $ 
\end{itemize}
\section{Average Velocity}
\begin{itemize}
\item $ average speed = \dfrac{distance \; traveled}{time \; elapsed}$
\item $ average velocity = \dfrac{displacement}{time \; elapsed}$
\item 
\end{itemize}
\section{Instatinius velocity}
\paragraph{The instatinius velocity is the average velocity, in the limit as the time intervl becomes infintesimall short \\ average velocity $ v = \dfrac{ \bigtriangleup x}{\bigtriangleup t}$ \\ instatinius velocity $ v = \lim_{\bigtriangleup t \to 0} \dfrac{\bigtriangleup x}{\bigtriangleup t}$ \\ average acceleration = $ \dfrac{change \:  of \: velocity}{time \:  elapsed}$ }
\section{Acceleration}

\begin{itemize}
\item Acceleration is a vector, although in one dimensional motion we only need the sign.
\end{itemize}
\pagebreak
\section{Motion at constant acceleration}
\begin{itemize}
\item the average velocity of an object during a time interval t is \\ $ v = \dfrac{x -x_{0}}{t-t_{0}} = \dfrac{x-x_{0}}{t}$
\item the acceleration assumed constant is  $ a = \dfrac{v-v_{0}}{t}$
\item Re-arrange: $ v = v_{0} + a*t$
\item Supose the partical starts at some position $x_{0}$ 
\item  If we knew  average velocity v, we could calculate $ x=x_{0} + v *t $
\item now for constant acc the avg velovity is $ v = \dfrac{v_{0}+v}{2}$
\item $ x = x_{0} + v_{0} t + \dfrac{1}{2}at^{2}$ formula for constant acceleration
\item constant acc problems know them \\ $v = v_{0} + at \\x = x_0 + v_0 t + \dfrac{1}{2}at^2 \\ v^2 = v_{0}^2 + 2a( x- x_0) \\ v = \dfrac{v+ v_0}{2}$ 

\end{itemize}
\section{solving problems }
\begin{itemize}
\item ex$\colon $ 
\end{itemize}
\section{falling objects}
\begin{itemize}
\item all object fall about the same acceleration due to graviy ignoring air resistace and other friction forces 
\item 
\end{itemize}

\chapter{first test}
\section{}
\begin{itemize}
\item airplane 
\end{itemize}
\end{document}