\documentclass[10pt,letterpaper]{report}
\usepackage[margin=0.75in]{geometry}
\usepackage{mathtools}
\usepackage{siunitx}
\usepackage{amsfonts}
\usepackage{amssymb}
\author{matt baird}
\title{math notes }
\begin{document}
\maketitle
\chapter{•}
\chapter{}
\section{}
\begin{itemize}
\item A set S is a collection of items $ \{2,chair,imortality , \{1\}\}$
\item 2 E S denotes that 2 is a member of set S. $"xES" \\"xED"$
\item Important sets \\ IR = set of all real numbers \\ Z = set of ints \\ Q = set of rational numbers \\ N = set of natural numbers
\item Def y A  B are sets, and every element of A is inside B , then A is a subset of B  $A \leq B$
\item $N\leq Q\\N\leq R \\N\leq Z \\Z \leq Q\\Z \leq R \\Z not \leq N \\ Q \leq R $
\end{itemize}
\chapter{Set Theory}
\section{Predicate and quantified statments}
\begin{itemize}
\item  Def a predicate is a sentence that contains a finite number of variables and that becomes a statement when specific values are substituted for the variable \\$P(x)="x \geq 2" \\ P (1) = 1 > 2" true \\P(1) = "1>2" false$
\item The domain of the predicate is the set of all values that may be substituted for  a variable. \\ $D = R \\D = \{1,2,3\} \\ D = Z^+$ (set of positive ints)
\item The truth set of P(x) is the set of elements in D such that P(x) is true.\\ $\{x \epsilon D \colon P(x)\} \\ \{x=1,2,3 \colon x > 2\} = \{3\} $
\item Ex finding the truth set  \\ $ Q (n) = $"n is a factor of g" \\ a) $ D=Z^+ $\\ \{ $n \epsilon Z^+ \colon $ n is a factor of 8 \}  \\ $ = \{1,2,4,8\} $ \\ b) $ D= Z \\ \{n \epsilon Z \colon n $ divides 8 evenly $\}$ \\$ = \{1,2,4,8,-1,-2,-5,-8\}$ 
\item Def The symbol $ \forall $ denotes " for all" (for every, for any) and is called the universal quantifier. \\ ' all humans are mortal" \\ "$\forall$ humans h , h is mortal" \\ define H = all humans " $ /forall h \epsilon H $, h is mortal " 
\item Def A universal statement has the form " $ \forall x \epsilon D, Q(x)$" where D is  a domain and Q(x) a predicate. \\ The statement is true if and only if Q(x) is true for each $x \epsilon D $. The statement is false id there is at least one $x \epsilon D $ such that Q(x) is false.
\item $\forall x \epsilon D, x^2 \geq  x \\ a) D = \{1,2,3,4\} \\ 1 \geq 1,4 \geq 2 , 9 \geq 3 , 16 \geq 4 \\ $ therefore the statement is true. \\ b) $ D=R \\ let x = .5 $counter example$ \\ $then $ x^2 = .25 \\ .25 < .5 $
\item Def the symbol "$\exists $" denotes " there exist" ( there is one ... , there are some..) and is called the existential quantifier. \\"there is a cat on the fridge' \\ $ C = \{all \; cats\} \\ \exists x \epsilon C , $ c is on the fridge. 
\item def an existential $" \exists x \epsilon D$ such that $ Q(x)"$ \\ "$ \exists x \epsilon D $ such that Q(x)" \\ is true when at least one x in D makes Q(x) true \\ is false when every $ x \epsilon D $ makes Q(x) false. \\ Consider $ \exists m \epsilon Z^+ $ such that $ m^2 = m  \\  \dfrac{m^2}{m}  = \dfrac{m}{m} \Rightarrow m = 1 $ \\ which in words is " there exits a positive int such that is is equal to its square." \\ "some positive integer equals its square"
\item " no dogs have wings" \\ All dogs don't have wings" \\ D = a set of dogs \\$ \forall d \epsilon D ,$ d doesn't have wings. \\ all, every, each ,none, no ; are universal words \\ at least one , some, there is / exists; are existential words 
\item Universal conditional statements \\ $\forall x , if P(x) then Q(x) \\ \forall x R , if x > 2, then x^2 > 4 \\ Ex \colon $ No sloths are fast. \\ $ \forall s \epsilon U ,$ if s is a sloth , then s is not fast. $\equiv  \forall s \epsilon S , $ s is not fast \\ P(x) = s is a sloth" \\ $ \{ x \epsilon u $ , if P(x) , then Q(x) \\ $ \{ x \epsilon u \colon P(x)\} \colon = D = \{sloths\} /equiv \forall x \epsilon D , Q(x).$ 
\item Ex $ \forall x \epsilon R , if x \epsilon Z, then x \epsilon Q \equiv  \forall x \epsilon Z , x \epsilon Q . $ \\ N $ \colon $ check 1,2,3 \\ Z $\colon $ check negative \\ Q $ \colon $  check $ \dfrac{1}{2}, \dfrac{3}{4}$ \\ R $ \colon $  check $ \pi , e , \sqrt{2} $ 
\item Ex there exits a number that is both even and prime  \\ $\exists n $ such that $P(n) \wedge E(n).$ \\ $\forall $ a prime number such that E(n) \\ $\forall $ an even number such that P(n). \\ " implicit quantification" \\ y a number is an int , then ... \\ $\forall n \epsilon Z $ \\ indefinite article "a" \\ "24 can be written as the sum of two ints" \\ $\forall m, n \epsilon Z $ such that $m + n = 24$ \\ $P(x) \Leftrightarrow Q(x). $\\ $ \{x \colon Px(x)\} = \{ x \colon Q(x) \}  $ 

\item homework 3.1 2,3,9,11,13,19
\end{itemize}
\section{Negations of quantified statments}
\begin{itemize}
\item $ ~( \forall x \epsilon D , Q(x)) \equiv \exists x \epsilon D , \; such \; that \; ~ Q(x)$ \\ $~(\exists x \epsilon D \; such \; that \; Q(x)) \equiv \forall x \epsilon D, \; ~Q(x).$ \\Negate "some dogs are bad dogs" \\ " all dogs are god dogs" \\ $~( \forall x \; if \; P(x) \; then \; Q(x)) \equiv \exists x \; such \; that \; P(x) \; and \; ~Q(x)$ \\ $ negating \; \forall \\ gives \; \exists \\ negating \; \exists \\ gives \; \forall $
\pagebreak
\item Universal statments that are vacuously true \\ Consider an empty jar \\"all the grapes in the jar are red" \\ $ \forall g \epsilon J$ \\ negating that = \\ "some grapes in the jar are not red" \\ $ \exists \; g \epsilon J \; Such \; that \; g \; is \; not \; red.$
\item Universal Conditional \\ $\forall x \epsilon D , \; if P(x)\; then \; Q(x) $ 

\end{itemize}
\section{Statments sith multiple quantifiers}
\paragraph{from quiz \\ any interger equals twice some interger \\ which is "$\forall n \epsilon z $}
\begin{itemize}
\item "everyone has someone to love" \\ "for every person, there is another person that they love " \\ $ \forall p \epsilon H , \exists r \epsilon H \; such \; that \; p \; loves \; r.$ 
\item $ ~( \forall x \epsilon D , \exists y \epsilon E $ such that P(x,y)) \\ $ \equiv \exists x \epsilon D \; such \; that \; \forall y \epsilon E, \; ~P(x,y)$ \\ $ ~(\exists x \epsilon D \;; such \; that  \; \forall y \; in \; E, \; P(x,y)) $ \\ $ \equiv \forall x \epsilon D , \exists y \epsilon E \; such \; that \; ~P(x,y). $
\end{itemize} 
\section{arguments with quantified statments}
\begin{itemize}
\item Rule of universal instantiation if a property is true of everything is a set, it is true for any partculur member of the set.
\item hello
\end{itemize}

\chapter{4}
\section{4.1}
\section{4.2}
\begin{itemize}
\item hello
\end{itemize}
\section{4.3}

\paragraph{from quiz \\ 1 for all ints m if m is even then $3m + 5 $ is odd \\ $m = 2k $ for some $ k \epsilon Z$ //$3m + 5 = 3(2k) + 5 = 6k + 5$ \\want $ 6k + 5 = 2 j + 1 $ for some $j \epsilon Z$ find the j \\ $6k + 5 = 2 j + 1 $  $ 6k + 4 = 2j $  $ 3k + 2 = j$ \\set $j = 3k + 2$ since $ k \epsilon Z , j \epsilon Z$ \\ we have $ 3m + 5 = 2 j + 1 $ for $ j \epsilon Z$  \\ 2 for all ints a,b,and c if a|b and  a|b then a|(b-c)\\ $a|b \Leftrightarrow b = ar$ for some $ r \epsilon Z$ \\ $a|c \Leftrightarrow c = as $ for some $ s \epsilon Z $ \\ $ b-c = ar- as ==a(r-s) $\\ since $ r, s \epsilon Z, r-s \epsilon Z $\\ $ a|(b-c)$}
\section{Division into cases and the quotient-remainder theorem}
\begin{itemize}
\item (quotient-remainder)  \\ given any interger n and interger $d > 0 $ , there exist unique interger q,r such that $n = d q + r $ where $0 \leq r < d $
\item given $ n = q d + r $ we say " n div d" is the integer quotient q obtained when n is divided by , and " n mod d" is the integer remainder obtained when n is divided by d.
\item representations of integers \\ we can prove all integers are even or odd using the QR \\ given any n , $6+ d = 2 $ then $ n = 2 q + r $ where $0 \leq r < 2 $ \\ so $ r = 0 $ or $ r = 1 $ \\ two cases \\ $ n = 2 q + 0$ or $ n = 2q + 1$ \\ but $q \epsilon Z $ \\ so n is even or n is odd
\item prove that any two consectitive integers have opposite parity ( that is one is even and one is odd). \\ let m ,$ m+ 1$ be two consective integers \\ case 1 \\ m is even therefore  $ m = 2 k = 2 k + 1$ for the same $k \epsilon Z $ therefore $m+1$ is odd \\ case 2 \\ therefore  $ m = 2 k + 1 $ for some $ k \epsilon Z $ then $ m + 1 = 2 k + 1 + 1 \\ = 2k + 2 \\ 2(k+1) $ \\ therefore $ (k + 1 ) \epsilon Z $ therefore  $m +1 $ is even 
\item can have more than 2 cases
\item show that any integer n can be written in one of the four forms \\ $ n = 4q \\ n = 4q+1 \\n = 4 q + 2 \\ n = 4 q + 3 $ \\ for some $q \epsilon Z $ \\proof \\ QR - ther $ 6 + d = 4 $ then for all integers , n , $ n = 4 q + r $ where $ 0 \leq r < 4 $ \\ four cases $ r = 0 , 1 , 2 , 3 \\  n = 4q \\ n = 4q+1 \\n = 4 q + 2 \\ n = 4 q + 3 $ 

\end{itemize}
\section{•}
\begin{itemize}
\item 
\end{itemize}
\chapter{Sequences and math induction}
\section{sequences}
\begin{itemize}
\item def a seqence ins a function whose domain is rither all the intergers or all the intergers greater that or equal to an integer.
\item $a_0,a_1,a_2,a_n $ \\ an element $a_k $ ( " a sub k " ) is called a term of the sequence and k is its index 
\item when the sequence donsnt have a final term it is a n infinite sequence 
\item Explicit formulae for sequences \\ $ a_k = \dfrac{k}{k+1}$ for all intsegers $ k \leq 1 $ \\ $ a_ 1 = \dfrac{1}{2} $ \\ $ b_i = \dfrac{i-1}{i}  $ for all ints $ i \leq 2 $ \\ $b_2 = 1/2 $ \\ $ b_3 = 2/3 $ \\ $ b_ 4 = 3/4$ etc \\ $ a_k = b_i $ where $ k = i - 1 $ 
\item alternating sequences \\ $ c_j = (-1)^j \;	for \; j \leq 0 $ \\ $ c_0 = (-1)^0 = 1 $ \\ $ c_1 = (-1)^1 = -1  $ \\ $ c_2 = (-1)^2 = 1 $ \\ $ c_3 = (-1)^3 = -1 $ \\ example \\ $ 1, -1/4, 1/9 , -1/16, 1/25, -1/36$ \\ 1 in numerator \\ signs alternate \\ perfict squares in denominator \\ list$ k = 1,2,3,etc$ for each term \\$ a_k = \dfrac{(-1)^(k+1) }{k^2} $ \\ positive $a_k $ when $k$ is odd  
\item summation notation \\ want  to take the sum of all the elements in a sequence ($ a_1 + a_2 + a_3 + ...$)  \\ $\sum\limits_{ k=1}^n a_k $ \\ let $ a_1 = -2 , a_2 = -1 , a_3 = 0 , a_4 =1 , a_5 = 2 $ \\ compute \\ $ \sum\limits_{ k = 1}^5 ak = a_1 + a_2 + a_3 + a_4+a_5 = -2, + -1 + 0 _ 1 + 2 = 0  $\\ $\sum\limits_{k=2}^2 a_k = a_2 = -1$ \\ $ \sum\limits_{ k=1}^2  a_2k = a_2(1) + a_2(2) = a_2 + a_4 = -1 + 1 = 0 $ \\ $\sum\limits_{k=1}^5 k^2 \; or \; \sum\limits_{i=0}^8 \dfrac{(-1)^i}{i + 1} $ \\ $ \sum\limits_{i=0}^n  \dfrac{(-1)^i}{i+1} = \dfrac{(-1)^0}{0+1}  + \dfrac{(-1)^1}{1+1}  +\dfrac{(-1)^2}{2+1}  +\dfrac{(-1)^3}{3+1}  + ... + \dfrac{(-1)^n}{n+1}  $ \\ $ = 1-1/2+ 1/3+ ... + \dfrac{(-1)^n}{n+1} $ \\ $ 1/n + 2/ (n+1) + 3/ (n+3 ) + ... + (n+1)/ (2n) $\\ the general term $ a_k = (k+1)/(n+k) $   $2n = n+ n$ \\the ints range from $(k=0)$ to $(k=n)$ \\$ \sum\limits_{k=0}^n \dfrac{k+1}{n+k} $
\item telescoping sum \\ $ \sum\limits_{k=1}^n \dfrac{1}{k(k+1)} $ \\ $ 1/k - 1/(k+1) = (k+1)/(k(k+1)) = (k)/(k(k+1)) = 1/(k(k+1))$ \\ $ = \sum\limits_{k=1}^n \dfrac{1/k}{1/(k+1)} $ $= ((1/1)-(1/2)) + ((1/2)+(1/3)) + ((1/3)-(1/4)) + ... + ( (1/(n-1)) -(1/n))+((1/n)-(1/(n+1)))$ 
\item product notation \\ $\prod\limits_{k=1}^n a_k = (a_1)(a_2)(a_3)...(a_n) \\ \prod\limits_{k=1}^5 k = 1*2*3*4*5 =120 $ \\ $ \sum\limits_{k=1}^n a_k +  \sum\limits_{k=1}^n b_k = \sum\limits_{k=1}^n (a_k + b_k) \\ \sum\limits_{k=0}^n a_k = a_0 + \sum\limits_{k=1}^n a_k $ \\ 2  $ c* \sum\limits_{k=1}^n a_k = \sum\limits_{k=1}^n a_k * c $ \\ $(\prod\limits_{k=1}^n a_k )(\prod\limits_{k=1}^n b_k)= \prod\limits_{k=1}^n (a_k * b_k)$
\item factorials \\$n! = n( n-1) (n-2) ... 3 * 2 * 1 = n(n-1)!$ \\ $n!= \prod\limits_{k=1}^n k $ \\ we define $0! =1$ $ (8!)/(7!) = (8*7!)/(7!) = 8 $ \\ $ n,r \epsilon Z $ $0 \leq r \leq n$ 

\end{itemize}
\paragraph{homework \\section 5.1\\ 3,12,19,31,40,49,55,65,75}
\paragraph{test october 9th chapters 2-5 \\ defs, proofs 4 questions 3 to do and bonus }
\section{Math induction}
\begin{itemize}
\item priciple of math induction \\ let P(n) be a property defined for intergers n and let a be a fixed int  \\ suppose the following to be true \\ P(a) is true \\ For all ints $k \geq a, $ if P(k) is true, then $P(k+1)$ is true. \\ then the statment $\forall ints n \geq a, P(n)$ is true
\item method of proof by math induction \\ show that P(a) is true \\ suppose P(k) is true for $ k \geq a$ ,k use this to show $ P (k+1)$ true 
\item example \\ $ 1+2+3+...+n = \dfrac{n(n+1)}{2}$ for all ints $n \geq 1$ which is P(n) \\ 1 show P(1) is true $n=1$ \\ $1 = \dfrac{1(1+1)}{2}= \dfrac{2}{2} = 1$ \\ 2 assume P(k) is true \\ $ \sum\limits_{j=1}^k j = \dfrac{k(k+1)}{2}$ \\ show $P(k+1)$  $ \sum\limits_{j=1}^(k+1) j = \sum\limits_{j=1}^(k) j + (k+1)$ = $ \dfrac{k(K+1)}{2}+k+1(\dfrac{2}{2}) = \dfrac{k^2 + k + 2k + 2}{2} = \dfrac{k^2 + 3k+2}{2}  = \dfrac{(k+1)(k+2)}{2} $ so $P(k+1)$ is true

\end{itemize}
\section{more math induction}
\par{1 showP(a) true \\ 2: Assume P(k) is true for some $ k \geq a $ use this to prove $P(k+1)$ is true }
\par{For all intergers $ n \geq 0, 2^(2n) - 1 $ is divisible by 3 \\ pf: show that P(0) is true \\ $ n =0: 2 -1 $ is divisible by 3 \\ $ 2^0 -1 = 1 - 1 = 0 . 0^n = 3^d * 0^k $ \\ so 0 is divisible by 3  \\ 2. Assume p(k) iis true for $ k \geq 0$ \\ know : $2^(2k) -1 $ is divisible by 3 for $ k \geq 0 $ \\ So $2^(2k) - 1 = 3 r $ for some $ r \epsilon Z$ \\$ 2^(2(k+1)) -1 = 2^(2k+2) -1 = 2^2k * 2^2 -1 = 4 * 2^2k -1 = 3 * 2^2k + 2^2k -1 = 3 * 2^2k + 3r = 3(2^2k + r) $ \\ let $s = 2^2k + r $ since $k,r \epsilon Z , s \epsilon Z $ Then $2^2(k+1) -1 = 3s , s \epsilon Z $ So $2^2(k+1) $ is divisible by 3 Thus $ P(k+1) $ is true So $P(n) $ is true for all integers $ n\geq 0 $ } 

\par{\textbf{Prove an inequality} \\ For all integers $n \geq 3 ,, 2n + 1 <2^n $ \\ show that $P(3)$ is true: $n =3 : 2(3)+1 = 7 2^3 =8. 7<8 $ \\ assume $2k+1< 2^k \; for \; k \geq3$ then $ 2( k+1)+1 = 2k + 2 + 1 = (2k+1)+2 < 2^k +2'  $ we know $ 2< 2^n \forall n > 1$\\ remind: if $ a <b \;then \; a+c < b+c $ \\ In particuler, for $ k\geq 3 , 2 < 2^k$ \\then $2(k+1)+1 < 2^k + 2 < 2^k + 2^k $ Now $ 2^k + 2^k = 2 * 2^k = 2^(k+1) $ \\thus $2(k+1)+1 < 2^(k+1) $}
\par{define a sequence $a_1,a_2,a_3,... $ as follows : $ a_1 =2 a_k = 5 * a_k-1 , \forall k \geq 2 \\(a_2 = 5a, = 5 * 2 = 10 ) $ prove that for $n \geq 1 , a_n = 2 *5^(n-1) $ \\ pf: $ P(1): a_1 = 2*5^(1-1) (check) \; a_1 = 2. 2*5^(1-1) = 2*5^0 =2 $ so P(1) is true \\ Assume P(k) is true. then $ a_k = 2* 5^k-1 \; for \; k \geq 1. a_k+1 = 5 a_k = 5 * 2 * 5^k-1 = 2*(5*5^k-1 ) = 2* 5^k-1+1 = 2*5^k \; So \; a_k+1 = 2*5^k \; So \; P(k+1) $ is true }
\pagebreak

\par{\textbf{MidTerm }\\ write the definitions for even, odd, prime, composite, divisibility  \\ ex: even $ h = 2k $ for some $ k \epsilon Z $  \\ \\ truth tables for $ p \Rightarrow q , p ^ q , p \vee q $ \\ \\ determine whether an argument is valid \\ indentify the major and minor premised the conclusion and any critical rows \\ .\\ T/F ch 2 thru 3 : negations of logical statments ; contrapositives of logical statments; valid and invalid argument forms ; translating between formal and informal language pg 121 \\ .\\ four proofs pick three 2 are direct proofs pg 180  \\ 1 is by contradiction or contraposition pg 198 \\ 1 by math induction pg 244\\ \\ memorize all the things}
\linebreak
\par{5.3 homework \\ 6,8,19,24 }
\section{strong math induction}
\par{ Let P(n) be a property defined for integration, and let a and b be fixed ints with $a \leq b$ \\ Supose the following statments are true \\ 1 $P(a) , P(a+1) , ... , P(b) $ are all true \\ 2 For ant in $ k \geq b$ if P(iP is rue is true for all $ i, a\leq i \leq k , $ then $ P(k+1) $ is true \\ then P(n) is true for $ n \geq a $ }
\par{ Method  \\ 1 Prove $P(a) , P(a+1) , ... , P(b) $ are true  \\ 2 Suppose for all i such that  $a\leq i \leq k $ P(k) is true and use this to show $ P(k+1) $ is true }
\par{ example : \\ Any int $ n > 1 $ is divisible by a prime number. \\ P(n) : " n is divisible by a prime number " "$ \exists \;prime\; p\;such \; that \; p|n. $" \\ Show P(2) is true \\ 2 is a prime number and 2|2 }
\par{ Suppose that for some $ k \geq 2 , P(i) $ is true for $ 2 \leq i \leq k $ \\ Want to show : $k+ 1 $ is divisible by a prime \\ Case 1 $k+1 $ is prime. Then $ k+1 | k+1 $ so $ \exists	 p $ such that $ p|k+1$ \\ case 2 ; $ K=1$ is composite then $ \exists a,b \epsilon N  $ such that $ k+1 = ab. $  $ a < a < k+1 $ and $ 1 < b < k+1 $ then $ 2 \leq a \leq k $ so P(a) is true  }
%\section{set theroy}
\chapter{Set Theory }
\paragraph{6.1 definitions and the elements method of proof}{if S is a set and P(x) is a property that elements of S may or may not satisfy, we define $ A={x \epsilon S | P(x)}$ "the set of all x in s such that P(x) is true" $N = { n \epsilon Z : n > 0} \\ N \leq Z $ \\Subsets : Def : $ A \leq B $ is $ x \epsilon A \; then \; x \leq B $ Negation : $ A \nleq B \; if \; \exists x \; such \; that \; x \epsilon $
\paragraph{element method of proof} To prove a set is a subset of another let x,y be sets Prove $ x \leq y :$ \\ 1) suppose $ x \epsilon X $ is a particular but arbatary element of X \\ 2) show that $ x \epsilon y$
\paragraph{example}  $ A = {m \epsilon | m = 6r + 12 \; for \; some \; r \epsilon Z} \\ B = { n \epsilon Z | n = 3s \; for \; some \\; s \epsilon Z}$ \\ a) Prove $ A \leq B $  Let k be a particular but arbitary member of A . then $ k \epsilon Z $ and there exists $ r \epsilon Z $ such that $ k = 6r + 12 $  \\ Want : $ k \leq B \Leftrightarrow \exists s \epsilon Z $ such that $ k = 3s$ \\ $ k = 5r + 12 = 3(2r + 4) $ \\ Let $ s = 2r + 4 $ Since $ r \epsilon Z , s \epsilon Z $ So $ k = 3s, s \epsilon Z \; and \; k \epsilon Z $ \\ $ \therefore k \epsilon B \\ \therefore A \leq B$ \\ b) disprove $ B \leq A $ ( prove $ B \nleq A $) \\ $ \exists x \epsilon B $ such that $ x \notin A $ $ x \in B $ so $ x = 3s $ for some integer s \\ Let $ s = 1 $ then $ x = 3 $ if $ 3 = 6r +12 $ then $ 6r = -9 \; \; r = -3/2 \in Z \\ \therefore x \notin A \\ \therefore B \nleq A \; \blacksquare $
\paragraph{Set equality} $ A = B $ iff $ A \leq B and B \leq A $ \\ $ A = [ m \in Z | m = 3a $ for some $ a \in Z ] $ \\ $ B = [ n \in Z | n = 3b -3 $ for some $ b  \in Z] $ \\ 1) Prove $ A \leq B $ Let x$ x \in A $ then $ x \in Z $ and $ \exists a \in Z $ such that $ x = 3a $  S ince $ a \in Z , (b - 1 ) \in Z $ Let $ b-1 = a $ Then $ x = 3a = 3 ( b-1) \;x = 3b -3 , b \in Z $ So $ x \in B \\ \therefore A \leq B $ \\ Prove $ B \leq A : $ Let $ y \in B $
\par{homework 6.1 \\ 1,(a,c,e),2 , 3(a,), 5 }
\paragraph{home work 6.1 number 5} $ C = n \epsilon Z | n = 6r-5 $ for some $ r \epsilon Z$ \\$ D = m \epsilon Z | m = 3s+1 $ for some $ s \epsilon Z$ \\ a: $C\leq D$ Let $ n \epsilon C $ Then $ \exists r \epsilon Z $ such that $ n = 6r -5 $ \\ $ 6r -5 = 3 s + 1 \\ 6r-6 = 3s \\2r-2 =s $ \\ Let $ s = 2r-2$ since $ r \epsilon Z,s \epsilon Z $ \\then $ 3s= 6r -6 \; and\;3s + 1 = 6r-5$ then $ n = 6r -5 == 3s+1$//So $ n \epsilon D $ So $ C \leq D$ 
\paragraph{operations on sets} Let A, B be subsets of set U \\ 1. The union of A and B , $ A \cup B $ is the set of all element that are in A or B \\ 2. the intersection of A and B $ A \cap B $ is the set of all elements that are in A and B \\ the difference $ B -A $ is the set of all elements in B that are not in A \\  The complement of A , $ A^c$, is the set of all elements not in A \\ $ A \cup B = [ x \epsilon U | x \epsilon A \; or \; x \epsilon B ]$ \\$ A \cap B = [  x \epsilon U | x \epsilon A \; and \; x \epsilon B ] $ \\ $ B -A = [ x \epsilon U : x \notin A \; and \; x \epsilon B] $ \\ $ A ^c = [ x \epsilon U | x \notin A  ] $ 
\paragraph{t } $ ( a , b ] = [ x \in R | a < x \leq b ] $\\ $ [a,b] = [x \in R | a \leq x \leq b]$ \\ $ A = ( -1,0] = [ x R | -1< x <\leq o ] \\ B = [ 0,1) = [x < \in R | o \leq x < 1 ]$  \\$ A \cup B = [ x \epsilon R | -1 < x < 1] \\ A \cap B = [ x \in R | x = 0 ] = [0] $ \\ $ B -A = [ x \in R | 0 < x > 1 ] = ( 0,1)$ \\ $ A^c = [ x \in R | x \leq -1 or x > 0 ] = ( - \infty , -1 ] \cup ( o, \infty ) $ \\ $ [B_i]_{i \geq 1 }  $ is an infident squence of subsets of U  \\ $ \infty \cup_{i=1} \ B_i \ = [ x \in U | x \in B_i $ for some $ i \geq 1 ] $ \\$ \infty \cap_{i=1} \ B_i \ = [ x \in U | x \in B_i $ for all $ i \geq 1 ] $ \\ DEF We Define $ null $, the empty set ( the null set ) as the set containd no elements $ [o , 1] \cap [2 , 4 ] = null $ \\ Def A and B are disjoint if $ A \cap B = null$ \\ Def A, ,...,$A_n $ are mutually disjoint if $ A_i \cap A_j = null$ whenever $ i != j$
\par{homework. 6.1 10,11,19,21,24}
\paragraph{7.1 1,15,38,40,41}
\paragraph{7.2 one to one ontto and inverse functions}def $ F: X -> y$ is one to one if for all $x_1,x_2 \in X $ if $F(x_1) = F(x_2) $ then $ x_1 = x_2$ $ <=>\; if x_1 != x_2, \; then \; F(x_1) != F(x_2)$ \\ negation $ F : X -> Y $ is not one to one iff $ \exists x_1 , x_2 \in X $ such that $ F(x_1) = F(x_2) $ and $ x_1 != x_2 $  \\ to prove a function is one to one :
\end{document} 