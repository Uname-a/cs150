%!TEX program = lualatex
\documentclass[12pt,letterpaper]{article}
\usepackage{inputenc}
\usepackage{lmodern}
\usepackage{fontspec}
\usepackage{amssymb}
\setmainfont{times new roman}
\setlength\parindent{24pt}
\fontsize{11pt}{12pt}\selectfont
\setlength{\parskip}{\baselineskip}%
\renewcommand{\baselinestretch}{1.0}
\usepackage[margin=1in]{geometry}
\author{Matthew Baird}


\begin{document}
\paragraph{Matt Baird \\ COM 123-023 \\ September 29, 2014 }
\begin{center}
\textbf{Self-Critique}
\end{center}
\par{ I started out very poorly, I jumped right into the speech without a lead in. I got the attention of the audience by showing them the fancy computers that use what I was showing them. The way I tried to get the audience attention was only effective because of the subject matter, if anyone in the audience would have know about it even slightly it would have seemed boring and not attention grabbing. The preview statement was non existent and the thesis was weak. The thesis Ethos was enforced by the lack of notes , the self made props shown , and the subject matter. With the subject being a very unknown thing and me showing it in easy to follow and logical steps, allowed the audience to not be worried about if it was correct or something I didn't have any experience on. The transitions were lacking especially from the introduction to the main points, and from the main points to the conclusion. From the introduction to the main points the transition was lacking causing me to use verbal fillers to try get to the next part. The transition from the main plualatexoints the  conclusion was non-existent, and thus caused the ending to be sudden and unpredictable for the audience. The flow of the speech was shaky for the introduction and the conclusion, the main part of the speech was pretty well organized ,being a step by step tutorial. The introduction was short and didn't have a preview statement, and a very weak thesis. The conclusion was non existent.  }

\par{ The PowerPoint was a major help to the speech. It allowed the audience to follow the speech with their eyes especially when the props were small. The props were hard to see but the PowerPoint made up for that. I did move around too much between getting out of the way of the PowerPoint and grabbing the props off the table.  I spoke with enthusiasm and while on the quiet side I spoke in a dynamic enough tone that didn't bore the audience. There were parts of the speech that I was seemingly rambling on especially in the beginning of the speech.  I did use verbal fillers and some pauses that broke the flow of the speech. I used mostly  "umm" as a verbal filler and it was concentrated in the beginning  of the speech. I did stand in front of the projector and thus block the view of it for many people. The table was not in an optimal position so I was moving around a lot, so I could grab props for each of the steps. I didn't use any note cards and only looked at the PowerPoint so i could point to a specific picture or to go back to the right slide. }

\par{From this speech I have see that I need to work on a few things. The main thing I need to work on is on the introduction and the conclusion. What I am going to change for the introduction is that I need to have a strong attention getting device, then I can get to the stating of the thesis. A preview statement is very much needed and was absent in the speech. The conclusion is another major part of the speech I need to work on. It being absent in my speech caused the speech to not end well leaving a bad impression in the audience. I need to orient my self better in the future , placing myself and my props in a place that is facing most of the audience and not blocking anything  I still need to work on the verbal fillers and awkward pauses that break up my speech. I also picked a topic that caused me to run short in time and thus shortening the introduction and conclusion to make it fit, so in the future I will need to take that into consideration. I feel confident about me not using note cards but I may use one for future speeches to see if that cuts down on the verbal fillers. }
\end{document}